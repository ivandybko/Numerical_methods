
\documentclass[12pt, a4paper]{article}

\usepackage[utf8]{inputenc}
\usepackage[T1]{fontenc}
\usepackage[russian]{babel}
\usepackage[oglav,spisok,boldsect,eqwhole,figwhole,hyperref,hyperprint,remarks,greekit]{./style/fn2kursstyle}
\usepackage{tikz}
\graphicspath{{./style/}{./figures/}}

\usepackage{multirow}
\usepackage{supertabular}
\usepackage{multicol}
\usepackage{mathtools}
% Параметры титульного листа
\title{Прямые методы решения систем
	линейных алгебраических уравнений}
\lab{1}
\author{И.\,Е.~Дыбко}
\creator{С.\,И.~Тихомиров}
\supervisor{А.\,О.~Гусев}
\group{ФН2-51Б}
\date{2024}

% Переопределение команды \vec, чтобы векторы печатались полужирным курсивом
\renewcommand{\vec}[1]{\text{\mathversion{bold}${#1}$}}%{\bi{#1}}
\newcommand\thh[1]{\text{\mathversion{bold}${#1}$}}
%Переопределение команды нумерации перечней: точки заменяются на скобки
\renewcommand{\labelenumi}{\theenumi)}
\begin{document}
	
	\maketitle
	
	\tableofcontents
	\section{Описание использованных алгоритмов}
	\section{Ответы на контрольные вопросы}
	\begin{enumerate}
	\item \textbf{Каковы условия применимости метода Гаусса без выбора
	и с выбором ведущего элемента?}
	\item \textbf{Докажите, что если $\det A \ne 0$, то при выборе главного
	элемента в столбце среди элементов, лежащих не выше
	главной диагонали, всегда найдется хотя бы один элемент,
	отличный от нуля.}
	\item \textbf{В методе Гаусса с полным выбором ведущего элемента приходится не только переставлять уравнения, но и менять нумерацию неизвестных. Предложите алгоритм, позволяющий восстановить первоначальный порядок неизвестных.}
	\item \textbf{Оцените количество арифметических операций, требуемых
	для QR-разложения произвольной матрицы A размера n×n.}
	\item \textbf{Что такое число обусловленности и что оно характеризует?
	Имеется ли связь между обусловленностью и величиной
	определителя матрицы? Как влияет выбор нормы матрицы
	на оценку числа обусловленности?}
	
	\item  \textbf{Как упрощается оценка числа обусловленности, если матрица является:
	а) диагональной;
	б) симметричной;
	в) ортогональной;
	г) положительно определенной;
	д) треугольной?}
	
	\item \textbf{Применимо ли понятие числа обусловленности к вырожденным матрицам?}
	
	\item \textbf{В каких случаях целесообразно использовать метод Гаусса,
	а в каких — методы, основанные на факторизации матрицы?}
	
	\item \textbf{Как можно объединить в одну процедуру прямой и обратный ход метода Гаусса? В чем достоинства и недостатки такого подхода?}
	\item \textbf{Объясните, почему, говоря о векторах, норму $\| \cdot \|_1$ часто
	называют октаэдрической, норму  $\| \cdot \|_2$ "--- шаровой, а норму
	 $\| \cdot \|_{\infty}$ "--- кубической.}
	\end{enumerate}
\end{document} 