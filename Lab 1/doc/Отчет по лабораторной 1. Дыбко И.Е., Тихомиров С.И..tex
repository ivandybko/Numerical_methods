
\documentclass[12pt, a4paper]{article}

\usepackage[utf8]{inputenc}
\usepackage[T1]{fontenc}
\usepackage[russian]{babel}
\usepackage[oglav,spisok,boldsect,eqwhole,figwhole,hyperref,hyperprint,remarks,greekit]{./style/fn2kursstyle}
\usepackage{tikz}
\graphicspath{{./style/}{./figures/}}

\usepackage{multirow}
\usepackage{supertabular}
\usepackage{multicol}
\usepackage{mathtools}
% Параметры титульного листа
\title{Прямые методы решения систем
	линейных алгебраических уравнений}
\author{С.\,И.~Тихомиров}
\supervisor{Г.\,В.~Гришина}
\group{ФН2-41Б}
\date{2024}

% Переопределение команды \vec, чтобы векторы печатались полужирным курсивом
\renewcommand{\vec}[1]{\text{\mathversion{bold}${#1}$}}%{\bi{#1}}
\newcommand\thh[1]{\text{\mathversion{bold}${#1}$}}
%Переопределение команды нумерации перечней: точки заменяются на скобки
\renewcommand{\labelenumi}{\theenumi)}
\begin{document}
	
	\maketitle
	
	\tableofcontents
	\section{Описание использованных алгоритмов}
	\section{Ответы на контрольные вопросы}
	
	\begin{enumerate}
	\item \textbf{Каковы условия применимости метода Гаусса без выбора
	и с выбором ведущего элемента?}
	
		\textbf{Без выбора ведущего элемента:} Метод Гаусса может быть применен, если на всех шагах на главной диагонали не возникает нулевых элементов: $$a^{(i-1)}_{ii}\ne 0, i = 1,2,\ldots,n.$$

		 \textbf {С выбором ведущего элемента:} Метод с выбором ведущего элемента  применим всегда, когда матрица невырожденная ($\det A \ne 0$).
		 
	\item \textbf{Докажите, что если $\det A \ne 0$, то при выборе главного
	элемента в столбце среди элементов, лежащих не выше главной диагонали, всегда найдется хотя бы один элемент, отличный от нуля.}
	
		 Для любой невырожденной матрицы обязательно существует хотя бы один ненулевой элемент в каждом столбце среди элементов, которые находятся на главной диагонали или ниже ее. В противном случае хотя бы один столбец состоял бы из нулей, что привело бы к нулевому определителю, что противоречит условию ($\det A \ne 0$):
	

	\item \textbf{В методе Гаусса с полным выбором ведущего элемента приходится не только переставлять уравнения, но и менять нумерацию неизвестных. Предложите алгоритм, позволяющий восстановить первоначальный порядок неизвестных.}
	
	Создадим два массива linearr и columnarr, где изначально будет числовая последованность $i=1,2,\ldots,n$. При перестановке уравнений (строк) или смене нумерации неизвестных (столбцов) будем менять элементы в этих массивах. 
	
	
	
	\item \textbf{Оцените количество арифметических операций, требуемых
	для QR-разложения произвольной матрицы $A$ размера $n \times n$.}
	
	



	\item \textbf{Что такое число обусловленности и что оно характеризует?
	Имеется ли связь между обусловленностью и величиной
	определителя матрицы? Как влияет выбор нормы матрицы
	на оценку числа обусловленности?}
	
		Числом обусловленности $M_{A}=\|A_{-1}\| \|A\|$ называется числом обусловленности матрицы $A$ (и $A_{-1}$ в силу симметрии формулы). Оно характеризует, насколько сильно ошибка в данных может повлиять на решение задачи.
		
		Если матрица плохо обусловлена (большое число обусловленности), то матрица близка к вырожденной, что связано с малым значением определителя.
		Матрица с маленьким числом обусловленности близка к ортогональной или хорошо обусловленной.
		Норма матрицы влияет на оценку числа обусловленности: в зависимости от выбранной нормы $\|\cdot\|$ значение $M_{A}$ может различаться.
	
	\item  \textbf{Как упрощается оценка числа обусловленности, если матрица является:}
	\begin{enumerate}
		\item \textbf{ диагональной;}
		\item \textbf{ симметричной;}
		\item \textbf{ ортогональной;}
		\item \textbf{  положительно определенной;}
		\item \textbf{ треугольной?}
	\end{enumerate}

	\begin{enumerate}
		\item \textbf{ Диагональная матрица:} $M_{A}=\frac{\max(|a_{ii}|)}{\min (|a_{ii}|)}$
		\item \textbf{ Симметричная матрица:} оценка зависит только от собственных значений. Если матрица симметрична и положительно определена, то $M_{A}$ можно оценить через отношение наибольшего и наименьшего собственных значений.
		\item \textbf{ Ортогональная матрица:}$M_{A}=1$, так как $A_{-1}=A_{T}$ и $\|A\|=\|A_{-1}\|=1$
		\item \textbf{ Положительно определенная:} оценка зависит от собственных значений; чем больше разброс, тем выше число обусловленности.
		\item \textbf{ Треугольная матрицая:} число обусловленности зависит от отношения наибольшего и наименьшего диагональных элементов.
	\end{enumerate}
	
	\item \textbf{Применимо ли понятие числа обусловленности к вырожденным матрицам?}
	
		Для вырожденных матриц ($\det A = 0$) число обусловленности формально не определено, так как $A_{-1}$
		не существует. Однако, если матрица почти вырожденная, можно использовать псевдообратную матрицу $A_{+}$
		для оценки обусловленности.
	\item \textbf{В каких случаях целесообразно использовать метод Гаусса,
	а в каких — методы, основанные на факторизации матрицы?}
	
		Метод Гаусса эффективен для решения систем линейных уравнений с квадратными матрицами, если матрица не слишком плохо обусловлена.
		
		Методы факторизации предпочтительны, когда требуется решить несколько систем с одной и той же матрицей, но разными векторами правых частей. Они также более устойчивы при вычислениях с плавающей запятой и в случае плохо обусловленных матриц.
	
	\item \textbf{Как можно объединить в одну процедуру прямой и обратный ход метода Гаусса? В чем достоинства и недостатки такого подхода?}
	
		Можно объединить прямой и обратный ход метода Гаусса, используя модифицированную схему, где вычисления производятся непосредственно в ходе исключения. Это уменьшает количество операций ввода-вывода, но усложняет алгоритм и снижает его численную устойчивость.
	
	\item \textbf{Объясните, почему, говоря о векторах, норму $\| \cdot \|_1$ часто
	называют октаэдрической, норму  $\| \cdot \|_2$ "--- шаровой, а норму
	 $\| \cdot \|_{\infty}$ "--- кубической.}
	 
	 	Норма $\|\cdot\|_1$ называется октаэдрической, потому что геометрическое место всех точек вектора с такой нормой образует октаэдр.
	 	
	 	Норма $\|\cdot\|_2$ называется шаровой, потому что множество всех векторов с такой нормой образует сферу в евклидовом пространстве.
	 	
	 	Норма $\|\cdot\|_{\infty}$ называется кубической, потому что множество точек с такой нормой образует гиперкуб (или куб в трёхмерном пространстве).
	 
	\end{enumerate}
\end{document} 